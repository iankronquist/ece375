% template created by: Russell Haering. arr. Joseph Crop
\documentclass[12pt,letterpaper]{article}
\usepackage{enumerate}
\usepackage{anysize}
\marginsize{2cm}{2cm}{1cm}{1cm}

\begin{document}

\begin{flushright}
{\large
ECE 375 Assignment 1\\
Ian Kronquist
}
\end{flushright}

To verify my solutions for problems 2 and 3 I built a state machine to emulate
some of the microoperations. You can find my code at
http://github.com/iankronquist/ece375statemachine.

\bigskip

\begin{enumerate}
    \item 
    \begin{enumerate}
        \item The design discussed in class had three bits in the opcode field.
        $2^3 = 8$ so the maximum number of instructions is 8.
        \item The registers TEMP, MDR, and AC are all one word wide, or 16
        bits. Since there are only 4K or 4096 words in memory the PC and MAR
        only need to be able to uniquely address those words. $log(4096) = 12$,
        so the PC and MAR only need to be 12 bits long.  The IR only needs to
        be able to hold an instruction, including the indirection bit, so it
        will be just 4 bits long.
    \end{enumerate}

    \item $STA$ $(x)+$\\
    \begin{tabular}{l l}
    Execute Cycle\\
    \hline
        1 & $TEMP \leftarrow MAR, MDR \leftarrow M(MAR)$\\
        2 & $MAR \leftarrow MDR, AC \leftarrow MDR$\\
        3 & $MDR \leftarrow M(MAR), AC \leftarrow AC + 1$\\
        4 & $MAR \leftarrow TEMP$\\
        5 & $TEMP \leftarrow AC$\\
        6 & $AC \leftarrow MDR$\\
        7 & $MDR \leftarrow TEMP$\\
        8 & $M(MAR) \leftarrow MDR$\\
    \end{tabular}
    \item The fetch cycle will be the same as usual.\\
    \begin{tabular}{l l}
    Fetch Cycle\\
    \hline
        1 & $MAR \leftarrow PC$\\
        2 & $MDR \leftarrow M(MAR), PC \leftarrow PC + 1$\\
        3 & $IR \leftarrow MDR_{opcode}, MAR \leftarrow MDR_{address}$\\
    \end{tabular}
    \begin{enumerate}
        \item $DCA$ $ Y$ \\
        \begin{tabular}{l l}
        Execute Cycle\\
        \hline
            1 & $MDR \leftarrow AC$\\
            2 & $AC \leftarrow 0, M(MAR) \leftarrow MDR$\\
        \end{tabular}
        \item $JMS $ $Y$ \\
        For this instruction we take advantage of the PC counter's incrementer
        to avoid a trip through the accumulator. We could alter the fetch cycle
        so that we don't increment the PC, because we will immediately be
        overwriting it, but that may make the CU significantly more
        complicated. I believe this is a trade off between the power consumed
        by the incrementer and adding a more complex design to the CU.\\
        \begin{tabular}{l l}
        Execute Cycle\\
        \hline
            1 & $MDR \leftarrow PC$\\
            2 & $M(MAR) \leftarrow MDR, PC \leftarrow MAR$\\
            3 & $PC \leftarrow PC+1$\\
        \end{tabular}
    \end{enumerate}
    \item AVR Assembly\\
    \bigskip
    \begin{enumerate}[i]
    \item $MOV$  $R1,R28$\\
        \begin{tabular}{l l}
            Registers & \\
            \hline
            R0   & 01 \\
            R1   & 02 \\
            R2   & 1B \\
            R3   & 07 \\
            R4   & 01 \\
            X    & 0106 \\
            Y    & 0102 \\
            SREG & FF \\
        \end{tabular}
        \begin{tabular}{l l}
            Data Memory & \\
            \hline
            0100 & 01 \\
            0101 & BE \\
            0102 & 35 \\
            0103 & EC \\
            0104 & 48 \\
            0105 & 2D \\
            0106 & 04 \\
            0107 & 02 \\
        \end{tabular}

    \item $LD$  $R1,Y+$\\
        \begin{tabular}{l l}
            Registers & \\
            \hline
            R0   & 01 \\
            R1   & 05 \\
            R2   & 35 \\
            R3   & 07 \\
            R4   & 01 \\
            X    & 0106 \\
            Y    & 0103 \\
            SREG & FF \\
        \end{tabular}
        \begin{tabular}{l l}
            Data Memory & \\
            \hline
            0100 & 01 \\
            0101 & BE \\
            0102 & 35 \\
            0103 & EC \\
            0104 & 48 \\
            0105 & 2D \\
            0106 & 04 \\
            0107 & 02 \\
        \end{tabular}

    \item $LDI$  $R4,33$\\
        The hexadecimal representation of the decimal number 33 is 0x21. \\
        \begin{tabular}{l l}
            Registers & \\
            \hline
            R0   & 01 \\
            R1   & 05 \\
            R2   & 1B \\
            R3   & 07 \\
            R4   & 21 \\
            X    & 0106 \\
            Y    & 0102 \\
            SREG & FF \\
        \end{tabular}
        \begin{tabular}{l l}
            Data Memory & \\
            \hline
            0100 & 01 \\
            0101 & BE \\
            0102 & 35 \\
            0103 & EC \\
            0104 & 48 \\
            0105 & 2D \\
            0106 & 04 \\
            0107 & 02 \\
        \end{tabular}

    \item $MUL$  $R2,R3$\\
        \begin{tabular}{l l}
            Registers & \\
            \hline
            R0   & 00 \\
            R1   & BD \\
            R2   & 1B \\
            R3   & 07 \\
            R4   & 01 \\
            X    & 0106 \\
            Y    & 0102 \\
            SREG & FF \\
        \end{tabular}
        \begin{tabular}{l l}
            Data Memory & \\
            \hline
            0100 & 01 \\
            0101 & BE \\
            0102 & 35 \\
            0103 & EC \\
            0104 & 48 \\
            0105 & 2D \\
            0106 & 04 \\
            0107 & 02 \\
        \end{tabular}

    \item $ROL$  $R3$\\
        \begin{tabular}{l l}
            Registers & \\
            \hline
            R0   & 01 \\
            R1   & 05 \\
            R2   & 1B \\
            R3   & 0E \\
            R4   & 01 \\
            X    & 0106 \\
            Y    & 0102 \\
            SREG & FF \\
        \end{tabular}
        \begin{tabular}{l l}
            Data Memory & \\
            \hline
            0100 & 01 \\
            0101 & BE \\
            0102 & 35 \\
            0103 & EC \\
            0104 & 48 \\
            0105 & 2D \\
            0106 & 04 \\
            0107 & 02 \\
        \end{tabular}
    \end{enumerate}
\end{enumerate}

\end{document}
