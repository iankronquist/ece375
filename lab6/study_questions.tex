% template created by: Russell Haering. arr. Joseph Crop
\documentclass[12pt,letterpaper]{article}
\usepackage{anysize}
\marginsize{2cm}{2cm}{1cm}{1cm}

\begin{document}

\begin{titlepage}
    \vspace*{4cm}
    \begin{flushright}
    {\huge
        ECE 375 Lab 6\\[1cm]
    }
    {\large
         Timer/Counters
    }
    \end{flushright}
    \begin{flushleft}
    Lab Time: Monday Noon-2:00pm
    \end{flushleft}
    \begin{flushright}
    Ian Kronquist
    \vfill
    \rule{5in}{.5mm}\\
    TA Signature
    \end{flushright}

\end{titlepage}
\section{Introduction}
In this lab we continued using interrupts and also learned to use the timer. We learned about the various timer modes and what control registers needed to be set to control them.

\begin{enumerate}

\item If we had set the 8 bit timers in normal mode and manually it would have required significantly more code and hardware setup. The interrupts would fire far more often using additional CPU time during which the processor may not be able to respond to other interrupts. Overall, I prefer the approach we used in lab.

\end{enumerate}

\section{Difficulties}
This lab required lots of careful reading of the manual. Additionally, I initially tried to use EICRB instead of EICRA for my interrupts which caused trouble.

\section{Conclusion}
With pulse width modulation timers are an effective way to control adjustable tasks like motor speed. They are surprisingly useful and efficient.

\section{Source Code}

\begin{enumerate}
\item Definitions: Declare register and address aliases. Notably, choose and address named Level to hold the speed level counter.\\

\item Interrupt Vectors: Declare three interrupt vectors, reset, external interrupt 0, and external interrupt 1.\\

\item INIT: Initialize the stack pointer, set PORTB for outputs, enable interrupts, and configure EICRA to fire interrupts 0 and 1 on a falling edge. Configure the 8 bit timers using TCCR0 and TCCR2. Zero the location of the Level counter and the motors, for good measure.\\

\item MAIN: An infinite loop. Does nothing.\\

\item Decr \& Incr: First save the registers. Then read in the Level counter. If it is at one of the boundary conditions (0x0 or 0xf), jump to DecrDone or IncrDone respectively. This is so the counter does not overflow. If the counter is not in danger of overflowing, decrement (or increment) it, mask it with 0xf so that it does not interfere with the high bytes of PORTB, and write the new value to PORTB. Next read the current value of OCR0, and decrement (or increment) it by 16, and write the new value to both OCR0 and OCR2. Finally fall through to DecrDone or IncrDone.\\
\end{enumerate}
\begin{verbatim}

.include "m128def.inc"			; Include definition file

;***********************************************************
;*	Internal Register Definitions and Constants
;***********************************************************
.def	mpr = r16				; Multipurpose register
.def	mpr2 = r17				; Multipurpose register
.def	waitcnt = r18				; Wait Loop Counter
.def	ilcnt = r19				; Inner Loop Counter
.def	olcnt = r20				; Outer Loop Counter

.equ	Level = 0x1000

.equ	EngEnR = 4				; right Engine Enable Bit
.equ	EngEnL = 7				; left Engine Enable Bit
.equ	EngDirR = 5				; right Engine Direction Bit
.equ	EngDirL = 6				; left Engine Direction Bit

;***********************************************************
;*	Start of Code Segment
;***********************************************************
.cseg							; beginning of code segment

;***********************************************************
;*	Interrupt Vectors
;***********************************************************
.org	$0000
		rjmp	INIT			; reset interrupt

		; place instructions in interrupt vectors here, if needed

.org	$0002					; Beginning of IVs
		rcall 	Decr			; Reset interrupt
		reti

.org	$0004					; Beginning of IVs
		rcall 	Incr			; Reset interrupt
		reti


.org	$0046					; end of interrupt vectors

;***********************************************************
;*	Program Initialization
;***********************************************************
INIT:

		ldi		mpr, low(RAMEND)
		out		SPL, mpr		; Load SPL with low byte of RAMEND
		ldi		mpr, high(RAMEND)
		out		SPH, mpr		; Load SPH with high byte of RAMEND

		; Initialize Port B for output
		ldi		mpr, $00		; Initialize Port B for outputs
		out		PORTB, mpr		; Port B outputs low
		ldi		mpr, $FF		; Set Port B Directional Register
		out		DDRB, mpr		; for output

		; Initialize Port D for inputs
		ldi		mpr, $FF		; Initialize Port D for inputs
		out		PORTD, mpr		; with Tri-State
		ldi		mpr, $00		; Set Port D Directional Register
		out		DDRD, mpr		; for inputs

		; Initialize external interrupts
		; Set the Interrupt Sense Control to low level 
		; NOTE: must initialize both EICRA and EICRB

		; Don't use any of the higher interrupts
		ldi		mpr, (1<<ISC01)|(0<<ISC00)|(1<<ISC11)|(0<<ISC10)
		;ldi		mpr, 0x0
		sts		EICRA, mpr

		; Interrupts 0 and 1 should fire on a falling edge
		;ldi		mpr, (1<<ISC41)|(1<<ISC40)|(1<<ISC51)|(1<<ISC50)
		ldi		mpr, 0x0

		out		EICRB, mpr
		; Set the External Interrupt Mask

		; Enable interrupts 2 and 3, External interrupt requests 0 and 1
		ldi		mpr, 0x03
		out		EIMSK, mpr

		ldi		mpr,	(1<<WGM01)|(1<<WGM00)|(1<<COM00)|(1<<COM01)|(1<<CS00)|(1<<CS10)
		out		TCCR0,	mpr
		out		TCCR2,	mpr

		; Turn off motors.
		ldi		mpr,	0xff
		out		OCR0,	mpr
		out		OCR2,	mpr

		ldi		ZL,		low(Level)
		ldi		ZH,		high(Level)
		ldi		mpr,	0x0
		st		Z,		mpr


		sei



;***********************************************************
;*	Main Program
;***********************************************************
MAIN:
		; poll Port D pushbuttons (if needed)

								; if pressed, adjust speed
								; also, adjust speed indication

		rjmp	MAIN			; return to top of MAIN

;***********************************************************
;*	Functions and Subroutines
;***********************************************************


Decr:
	; Save registers
	push    mpr
	push    mpr2
	push    ZH
	push    ZL

	ldi		ZL,		low(Level)
	ldi		ZH,		high(Level)
	ld		mpr,	Z

	cpi	    mpr,	0x0
	breq    DecrDone

	dec		mpr
	andi	mpr,	0x0F
	st		Z,		mpr
	out		PORTB,	mpr



	in	    mpr,	OCR0
	ldi		mpr2,	16
	add		mpr,	mpr2
	out		OCR0,	mpr
	out		OCR2,	mpr

DecrDone:
	pop		ZL
	pop		ZH
	pop		mpr2
	pop		mpr
	ret


Incr:
	; Save registers
	push    mpr
	push    mpr2
	push    ZH
	push    ZL

	ldi		ZL,		low(Level)
	ldi		ZH,		high(Level)
	ld		mpr,	Z

	cpi	    mpr,	0x0f
	breq    IncrDone

	inc		mpr
	andi	mpr,	0x0F
	st		Z,		mpr
	out		PORTB,	mpr



	in	    mpr,	OCR0
	ldi		mpr2,	16
	sub		mpr,	mpr2
	out		OCR0,	mpr
	out		OCR2,	mpr

IncrDone:
	pop		ZL
	pop		ZH
	pop		mpr2
	pop		mpr
	ret


\end{verbatim}
\end{document}

\end{document}
